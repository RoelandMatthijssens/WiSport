\subsection{Deploy for Development}
When you use ruby a lot, it is recommended that you use something like the Ruby Version Manager\footnote{check https://rvm.io for more information}, or RVM for short. This will help you manage different ruby versions, and different GEM sets. RVM allows you to create separate sets of gems for each project you do. We will assume you have this setup or, in case you do not want to use rvm, know that it can interfere with other ruby projects.\\
To start the development on this project you will first have to check out the git repository using the following commando.
\begin{lstlisting}
git clone git://github.com/rikvanmechelen/web_info_sys_sport.git
\end{lstlisting}
This will create the project directoy. Now go into the wisport directory, which consequetly is the \textit{RAILS\_ ROOT}, the root of the Ruby on Rails project.\\
Since we use SQLite as DBMS, all you have to do is run:
\begin{lstlisting}
	rake db:migrate
\end{lstlisting}
Once this is done, you are free to poppulate the database: 
\begin{lstlisting}
	rake db:populate[10]
\end{lstlisting}
the 10 means that we will create 10 users.\\
By now you have a fully working rails application.
To start the rails server do:
\begin{lstlisting}
	rails server
\end{lstlisting}
All that rests, is pointing your favourite web browser to http://localhost:3000

\subsection{Deploy for production}
