\paragraph{}Ruby on rails is an opensource full-stack webframework. This means that Ruby on Rails, or Rails in short is used to tackle every aspect of webdevelopment. Rails runs on the general purpose programming language Ruby.
\subsection{Framework} % (fold)
\label{sub:Framework}
\paragraph{}The main benefit of using the Rails framework is the speed at which one can get a running web application. Almost no setup is required, and once the setup is complete, the paradigm of Convention over Configuration ensures that most of the website is developped quickly.
\paragraph{}Ruby on rails employs, like most webframeworks, a Model-View-Controller (MVC) architecture. By default all models in the application are mapped directly to a database. This makes it easy to have persistance while the programmer can only consern himself with the classes, and the logic of the application. Instead of how this data is saved to the database.
% subsection Framework (end)
\subsection{RubyGems} % (fold)
\label{sub:RubyGems}
\paragraph{}Another huge benefit of using Rails is the fact that is has an enormous comunity. This community creates exrta plug and play functionality for a rails application by means of rubygems. Rubygems (or gems in short) are self contained libraries that can be installed into a rails application.
% subsection RubyGems (end)
\subsection{Performance} % (fold)
\label{sub:subsection name}
\paragraph{}Ruby on Rails runs on the Ruby language, which is an interpretated language, instead of a precompiled alternative like the JavaBeans approach. This has ofcourse a significant performance drawback. And this is one of the main critics on Rails. However, since it hase been proven time and again, that rails applications do scale for big userbases. A decently deployed rails application can serve over 6000 requests per second. Which is twice as fast as a java beans applications deployed on an apache server with default configurations
% subsection subsection name (end)
\subsection{Why did we choose rails} % (fold)
\label{sub:Why did we choose rails}
\paragraph{}When we started the project we had a few frameworks to choose from. Our options were quickly limited to Ruby on Rails, Django, CakePHP or Drupal. Since non of the members of our team are particulary fond Java or the frameworks that come with it, these options were quickly dismissed. 
\paragraph{}Our experiance with Drupal in last year already made us reluctant to try it again. On top of that, already some of the other groups were using drupal.
\paragraph{}CakePHP is very opiniated, and the ease of development is strongly linked to the type of application one would create. Since stepping out of the Cake bounderies can be troublesome if we would try to add functionality that was not envisioned by the framework. On top of this Cake uses a CRUD based interface, which operates on data, while we prefer a REST interface which operates on resources.
\paragraph{}This left us with the choice between Rails and Django. Both frameworks are very simular. They both use a MVC architecture, they both use Object Relational Mapping (ORM) and they both use the principle of Don't Repeat Yourself (DRY). However the dealbreaker with Django was related to the templating system they employ. While Rails allows us to write HTML with embeded adhoc executed pieces of ruby code, django forces you into a template in which one can put variables calculated in advanced, but not ad-hoc. Because we wanted to create a user driven website, this ad-hoc calculation is crucial.
\paragraph{}This left us with ruby on rails. The added bonus that the majority of our group has already had some experience with Rails was also welcome.
% subsection Why did we choose rails (end)
