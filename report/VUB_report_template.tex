\documentclass[11pt,a4paper]{scrartcl}
% scrartcl.cls, scrreprt.cls, scrbook.cls ??? created from european point of view
% KOMA-script package required for these layouts
\usepackage[pdftex]{graphicx}
\usepackage[dutch]{babel}
\usepackage{palatino}
\usepackage{relsize}
\usepackage[absolute]{textpos} % places boxes at an absolute position - used for placing the logo's on the cover page
    \setlength{\TPHorizModule}{1mm}            % length of 1 unit horizontally
    \setlength{\TPVertModule}{\TPHorizModule}   % length of 1 unit vertically
    \textblockorigin{0mm}{0mm} % position from the top left from where positions are calculated
    \setlength{\parindent}{0pt}
%\usepackage{a4wide}                     % Iets meer tekst op een bladzijde
% the a4wide should not be used anymore. Use a4paper in the documentclass
\usepackage{amsmath}                    % Uitgebreide wiskundige mogelijkheden
\usepackage{amssymb}                    % Voor speciale symbolen zoals de verzameling Z, R...
\usepackage{makeidx}                    % Om een index te maken
\usepackage{url}                        % Om url's te verwerken
\usepackage{graphicx}                   % Om figuren te kunnen verwerken
\usepackage[small,bf,hang]{caption}     % Om de captions wat te verbeteren
\usepackage{xspace}                     % Magische spaties na een commando
\usepackage[latin1]{inputenc}           % Om niet ascii karakters rechtstreeks te kunnen typen
\usepackage{float}                      % Om nieuwe float environments aan te maken. Ook optie H!
\usepackage{flafter}                    % Opdat floats niet zouden voorsteken
\usepackage{listings}                   % Voor het weergeven van letterlijke text en codelistings
\usepackage[round]{natbib}              % Voor auteur-jaar citaties.
\usepackage[nottoc]{tocbibind}		% Bibliografie en inhoudsopgave in ToC; zie tocbibind.dvi
\usepackage{eurosym}                    % om het euro symbool te krijgen
\usepackage{textcomp}                   % Voor onder andere graden celsius
\usepackage{fancyhdr}                   % Voor fancy headers en footers
\usepackage[Gray,squaren,thinqspace,thinspace]{SIunits} % Om elegant eenheden zetten
% \usepackage{setspace}         % I use the command \baselinestretch to modify line spacings
% Volgend package is niet echt nodig. Het laat echter toe om gemakkelijk elektronisch
% te navigeren in je pdf-document. Deze package moet altijd als laatste ingeladen worden.
\usepackage[a4paper,plainpages=false]{hyperref}    % Om hyperlinks te hebben in het pdfdocument.


% De splitsingsuitzonderingen
\hyphenation{back-slash split-sings-uit-zon-de-ring}

%\bibpunct{(}{)}{;}{y}{,}{,}             % Auteur-jaar citaties -- zie natbib.dvi voor meer uitleg; niet echt nodig
% Het bibliografisch opmaak bestand.
% ZORG ERVOOR DAT bibliodutch.bst ZICH IN JE WERKDIRECTORY BEVINDT!!!
% \bibliographystyle{bibliodutch}

%\setlength{\parindent}{0cm}             % Inspringen van eerste lijn van paragrafen is niet gewenst.
% never use absolute sizes to modify parintents
%\renewcommand{\baselinestretch}{1.2} 	% De interlinie afstand wat vergroten.

\graphicspath{{afbeeldingen/}}               % De plaats waar latex zijn figuren gaat halen.

\makeindex                              % Om een index te genereren.

% De headers die verschijnen bovenaan de bladzijden, herdefinieren:
% deze headers mogen pas op de 2de pagina beginnen!

%\pagestyle{fancy}                       % Om aan te geven welke bladzijde stijl we gebruiken.
%\fancyhf{}                              % Resetten van al de fancy settings.
%\renewcommand{\headrulewidth}{0pt}      % Geen lijn onder de header. Zet dit op 0.4pt voor een mooie lijn.
%\fancyhf[HL]{\nouppercase{\textit{\leftmark}}} % Links in de header zetten we de leftmark,
%\fancyhead[HR]{\thepage}                % Rechts in de header het paginanummer.
% Activeer de volgende lijn en desactiveer de vorige om paginanummers onderaan gecentreerd te krijgen.
%%\fancyhf[FC]{\thepage}                  % Paginanummers onderaan gecentreerd.

% PDF specifieke opties, niet strict noodzakelijk voor een thesis.
% Is hetgeen verschijnt wanneer je in acroread de documentproperties bekijkt.
\hypersetup{
    pdfauthor = {Gaspard Lequeux},
    pdftitle = {Een Introductie tot het Zetsysteem LaTeX},
    pdfsubject = {Cursus LaTeX opgebouwd als typevoorbeeld voor het schrijven van een thesis.},
    pdfkeywords = {LaTeX, zetsysteem, thesis, eindwerk}
}


% Het volgende commando zou ervoor moeten zorgen dat er een witte ruimte wordt gelaten tussen
% elke paragraaf. Het zorgt ervoor dat er echter teveel witte ruimte komt boven en onder de
% verschillende titels, gemaakt met \section, subsection...
%%\setlength{\parskip}{0ex plus 0.3ex minus 0.3ex}

% Vandaar dat we expliciet aangeven wanneer we wensen dat een nieuwe paragraaf begint:
% \par zorgt ervoor dat er een nieuwe paragraaf begint en
% \vspace zorgt voor vertikale ruimte.
\newcommand{\npar}{\par \vspace{2.3ex plus 0.3ex minus 0.3ex}}

% Super en subscript
\newcommand{\supsc}[1]{\ensuremath{^{\text{#1}}}}   % Superscript in tekst
\newcommand{\subsc}[1]{\ensuremath{_{\text{#1}}}}   % Subscript in tekst

% Niew commando om vreemde taal weer te geven (hint: dit commando kan gebruikt
%   worden om latijnse namen, die ook cursief moeten staan, weer te geven.
\newcommand{\engels}[1]{\textit{#1}\xspace}
\newcommand{\engelsx}[1]{\index{#1}\textit{#1}\xspace}

% Niew commando om iets te benadrukken en tegelijkertijd in de index te steken.
\newcommand{\begrip}[1]{\index{#1}\textbf{#1}\xspace}

% Een nieuwe omgeving om algemene letterlijke tekst weer te geven.
\lstnewenvironment{lt}
    {
    \vspace{1.2ex plus 0.5ex minus 0.5ex}   % Beetje ruimte voor de letterlijke tekst
    \lstset{                                % Enkele opties:
        basicstyle={\small\tt},             % Iets kleiner en typmachine lettertype
        stepnumber=0,                       % De lijnen worden niet genummerd
        breaklines=true,                    % Als een lijn te lang is, wordt hij afgebroken
        basewidth={0.5em},                  % Breedte van een letter
        xleftmargin=1em}                    % Inspringing van de linker marge
    }
    {\vspace{0.9ex plus 0.5ex minus 0.5ex}  % Beetje ruimte na de letterlijke tekst
    }
% VUB templates use the verdana font. This is a sans serif font so we change the default font family to sans serif
\renewcommand{\familydefault}{\sfdefault}


\makeatletter
%\def\thickhrulefill{\leavevmode \leaders \hrule height 1pt\hfill \kern \z@}    % copied from somewhere else - no need for it and should be replaced by \newcommand...

\renewcommand{\maketitle}{
 \begin{titlepage}%

% faculteit: \small (9pt) x:41/y:25
\begin{textblock}{75.5}[0,0](41,28) % you need to keep this textblock before the VUB logo as it would otherwise be placed underneath the logo
   % \textblocklabel{Faculteit}
    \small{Faculty of Science}
\end{textblock}
% vub logo
\begin{textblock}{75}[0,0](15,12)
\textblocklabel{vub logo}
  \includegraphics[width=75.5mm]{VUB_logo.jpg}
\end{textblock}

\begin{textblock}{43}[0,0](157,229)
  \textblocklabel{vub schild}
  \includegraphics[width=43mm]{VUB_schild.png}
% Note: the color of this image should be 30% of black. I will create this image if I cannot find the command to apply alpha transparencies to an image.
\end{textblock}

% vertikale lijn: lengte: 247mm dikte hairline (0.25pt) - x: 38.5/y:25
\begin{textblock}{1}[0,0](38.5,25)
\textblocklabel{Vertical line}
    \vrule height 247mm
\end{textblock}

% eerste horizontale lijn: lengte: 161 hairline x:130/y:38.5
\begin{textblock}{161}[0,0](38.5,130)
\textblocklabel{first horizontal line}
    \hrule width 161mm
\end{textblock}

% 2de horizontale lijn: x:150/y:38.5
\begin{textblock}{161}[0,0](38.5,150)
\textblocklabel{second horizontal line}
    \hrule width 161mm
\end{textblock}

% datum: 10pt (\normalsize) - x:48/y:268
\begin{textblock}{75}[0,0](48,268)
\textblocklabel{Date}
  \normalsize{\@date}
\end{textblock}

% rapport naam: \Huge (eigelijk 30pt maar dat gaat niet) - x:48/y:100
\begin{textblock}{120}[0,1](48,123)         %  indien de titel meer dan 1 regel is schuift de titel naar boven
  \Huge{\@title}
\end{textblock}

% rapport code/ subtitle - \huge - x:48/y:137
\begin{textblock}{160}[0,0](48,133)         % use (48,133) for a 2-line title; use (48,137) for a 1-line title!
  \huge{Report project Web and Information Systems}
\end{textblock}

% auteurs \Huge - x:48/y:158
\begin{textblock}{160}[0,0](48,158)         %
  \Huge{\@author}                       %
\end{textblock}
\end{titlepage}%
\setcounter{footnote}{0}%
}

\makeatother


\author{Roeland Matthijssens, Rik Vanmechelen, Ayrton Vercruysse}
\title{WiSport}
\date{Academic year 2012-2013}

% list of possible special pages + title in english:
%\prefacename Preface 
%\refnamea References 
%\abstractname Abstract
%\bibnameb Bibliography
%\chaptername Chapter
%\appendixname Appendix
%\contentsname Contents
%\listfigurename List of Figures
%\listtablename List of Tables
%\indexname Index
%\figurename Figure
%\tablename Table
%\partname Part
%\enclname encl
%\ccname cc
%\headtoname To
%\pagename Page
%\seename see
%\alsoname see also
% if you want to change the title of these things, use something like this:
%\renewcommand*{\listfigurename}{Pictures}




%%%%%%%%%%%%%%%%%%%%%%%%%%%%%%%%%%%%%%%%%%%%%%%%%%%%%%%%%%%%%%%%%%%%%%%%%%%%%%%%%%%%%%%%%%%%%%
\begin{document}
%\frontmatter   % only for 'book', not for 'article'
\maketitle
% next 3 lines are preventing the next text from beginning on the titlepage.
% the reason is probably because I built up the titlepage with textboxes and there is 
% no normal text on it so creating a newpage doesn't work for starting on the next page.
\newpage
\vfil\null
\newpage
% everything from now on will be on the second page.
\tableofcontents
\newpage    %we want a new page after the table of content
\renewcommand{\baselinestretch}{1.5} 	% vanaf hier de interlinie afstand wat vergroten.
% regelafstand 1,5                      % indien je dit vanaf het begin doet is de titelpagina niet goed.
\small\normalsize                       % Nodig om de baselinestretch goed te krijgen.
%\mainmatter

\section{Introduction}
This document will describe the implementation and usage of WiSport. WiSport is website created within the scope of the course Web and Information Systems at the Vrije 
Universiteit Brussel. The goal of the website is to offer users a guide with with their personal training, aswell as making users able to do some training exercises 
together with some friends.


WiSport offers users the possibility to create exercises. These exercises can be exercises over time, distance or a number of repetitions. These exercises can be
shared with the WiSport community. Next to exercises users can also create training sessions. Training session contain a number of exercises that have to be completed
within this one session. Just as with the exercises these sessions can be shared with the WiSport community.


Shared exercises or sessions can be found by all WiSport users. These users can star these exercises or sessions. Each starred exercise or session gets added to the 
exercises or sessions of the user himselve. This means that from now one users can create new sessions containing some starred exercises.


Next to exercises and training sessions users can also create events. Events are moments on wich a training session will take place. Users can add to each event a place
where it will take place. After creating an event this event gets added to the personal calendar of the user.

When a session is done multiple times we can track our progress in doing this session, or on each individual exercise of the training session. The progression made is 
showed in clear graphs

In short is WiSport the ideal help for anyone willing to organize their training and share these trainings with other users.

\section{Ruby On Rails}
\section{Used API's}
subsubsectietekst
\section{Features}
\subsection{Login through social media}
Next to the tradtional possibility to sign up and login with an account made on our
website there's the possibility to log in with multiple versions of social media. We
have provided a log in with Google, Facebook and Twitter. When logging in with you

Google account the picture of you Google profile is copied on to your WiSport profile.

\subsection{Editting sessions/exercises/events}
After you have created an exercise, session or event you have to possibility to change
this event. In the case of an exercise you can edit your properties of this exercise.
A clear example of this could be if you have a distance exercise, in wich you run 5
kilometers. Chances are big that after half a year you don't even bother about running
5 kilometers, but would like to start running 10 kilometers in stead. If this is the case
you don't need to create a new exercise, being a running exercise of 10 kilometers, and
possible delete the old exercise of 5 kilometers. You can just edit the existing exercise
to a 10 km running exercise. As the graphs of your progress do not rely on this value,
but rely on the value you enter when doing an exercise this will not effect your progress
on this exercise. Within your progress graph you'll see the actual distances you ran.
\subsection{Publish exercises and sessions}
When a user creates an exercise or session and they are pleased with it they can share it whith the WiSport community. This is done by
pressing the Publish button on showed on each page of an exercise or session, showed within the list of own exercises, or even when 
creating an exercise or session you can immediatly choose to publish this exercise or session.

The published exercises can be starred by other users. As explained in the starred feature when starred a user can use an exercise
or session to create own sessions or events.
\subsection{Smart searchfields for exercises and sessions}
Within the list of all exercises and session some search form are implemented. You can search by name of the exercise/session or by
the creator. Next to this we also implemented a select box for the exercises. With this select box you can choose the type of exercises
that are getting displayed. This can help you whenever you are looking for a particular exercise to add to you own exercises. All these
searchforms can be combined together so that you can narrow your search down a long way.

The search forms are smart forms, this means you don't need to press search anywhere, but they dynamiclly change the list filtering out 
any results not matching the searchfield.

\subsection{Star sessions/exercises}
Every user has the possibility to publish their exercises or sessions. When published
these will apprear in the list of all exercises or sessions and other users will be able to
see these. When a user is browsing true this list they can star exercises or session which
look particular intresting to them. They can star these exercises or sessions and when
starred these will apear in personal excercises or sessions of the user.

This means that starred exercises can be used to create sessions and starred session can
be used to organise an event containing this session.


\subsection{Do a session}
Next to the possibilty to create a session an create an event around this session you
actually need to do a session. This means that even though you planned an event
today, you still have to say on the website you effectively done it. This is because
there's always a possibility somthing comes up, so you didn't do the event, but even
more, when doing a session you can enter some extra information. This can be some
additional notes, but, more importantly you can insert the actual value's you did for each
exercise. This means that if your exercise told you to run 10 kilometers, but you only
ran 8 you can enter this in the systems and this will take this into account when creating
your progression graph.
\subsection{Graphs tracking progress}
After you have completed a session and entered the actual numbers you performed to
each exercise graphs will be generated for earch exercise within this session. This gives
you the possibility to track your progression on a session, and even more you can check
the number times you completed an exercise, or the entire session.

Next to making graphs on the completion of exercises, you also get graphs on your
entire usage of the platform. For example on the login page you get a pie-graph which
shows you what percentage of what kind of exercises you made. This can be helpfull for
you to decide what kinds of exercise you can need more training in.

\subsection{Calendar view for events}
For events we make use of a clear calendar view. On this view you can see your planned
events, and from this view you can go to the actual page of this event, or delete or edit
this event. This gives you a clear view on your workload and makes it easier to plan
your sessions

\subsection{Weather}
On the events page before creating an event you can check the weather for your current
location. This gives you the possibility to plan your exercise session on a day the
weather is suitable. For example shedule your indoors exercises on a rainy day and plan
your outdoor activities on days it is less likely to rain.
\subsection{Share on Google+}
When you created a nice exercise or session on wich you are really proud you can
share this on you Google+ profile. Just as if you planned an event within the WiSport
platform you can choose to share this on your Google+ profile. This ensures you only
share this information with people you want it to be shared with. The idea behind this
is privacy, especially sharing on the internet when you are not going to be home isn't
always a good idea.
\subsection{Star sessions/exercises}
Every user has the possibility to publish their exercises or sessions. When published
these will apprear in the list of all exercises or sessions and other users will be able to
see these. When a user is browsing true this list they can star exercises or session which
look particular intresting to them. They can star these exercises or sessions and when
starred these will apear in personal excercises or sessions of the user.

This means that starred exercises can be used to create sessions and starred session can
be used to organise an event containing this session.

\subsection{Gravatar usage}
When people create an account or log in not using their Google account a picture is provided by Gravatar. This means people don't have 
to upload their profile picture to their WiSport account, but their pictures used by Gravatar is inserted in to their profile page.
\subsection{Scaleable interface}
Thanks to the use of Twitter Bootstrap the entire interface scales to the size of the screen it is displayed on. This means that when
the size of the screen is made smaller the entire website is scaled with it. Even when the size is to small to show the entire headerline
this line is reformed to a scrolldown box.

\subsection{Add location to an event by giving address}
Whenever an event is created the user can enter where this event will take place. This is done by just entering the address of the 
location. When the event is created on the event page a map will be shown on wich the location of the event is showed.
\subsection{Location my ip/ geoloc}
For the usage of the Weather API we make use of Location by ip. This means that you don't need to enter your location, but the location
of your current ip address is tracked.
\subsection{Find nearby events}
When going to the events page your can find all events planned on a map. This map is centralized around your curren location, so the 
closest events will be shown first. This gives you the ability to find all near events and if you please tag along on these.

By clicking on the markers more information is shown about this event. You can also go to the event page to get more information about
the event, and plan go along with this event.
\subsection{Database populator for testing purpose}
For testing purposes a database populator is created. This creator generates random dat, such as users, exercises, training sessions and
events. 

The use of this populator is to check the site's performances and acting with a higher number of users and data. This gives us the 
possibility to do some experiments with a lot of data. Especially for the creating the graphs this can be usefull.
\section{Tutorial}
\section{Deployment information}


% De bibliografie en de index
%\backmatter
\newpage        % we want the bibliography to start at a new page
\bibliography{bibliografie}
\newpage
\printindex                             % Om de index af te printen, niet bij een thesis.

\end{document} 