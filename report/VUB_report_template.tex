\documentclass[11pt,a4paper]{scrartcl}
% scrartcl.cls, scrreprt.cls, scrbook.cls ??? created from european point of view
% KOMA-script package required for these layouts
\usepackage[pdftex]{graphicx}
\usepackage[dutch]{babel}
\usepackage{palatino}
\usepackage{relsize}
\usepackage[absolute]{textpos} % places boxes at an absolute position - used for placing the logo's on the cover page
    \setlength{\TPHorizModule}{1mm}            % length of 1 unit horizontally
    \setlength{\TPVertModule}{\TPHorizModule}   % length of 1 unit vertically
    \textblockorigin{0mm}{0mm} % position from the top left from where positions are calculated
    \setlength{\parindent}{0pt}
%\usepackage{a4wide}                     % Iets meer tekst op een bladzijde
% the a4wide should not be used anymore. Use a4paper in the documentclass
\usepackage{amsmath}                    % Uitgebreide wiskundige mogelijkheden
\usepackage{amssymb}                    % Voor speciale symbolen zoals de verzameling Z, R...
\usepackage{makeidx}                    % Om een index te maken
\usepackage{url}                        % Om url's te verwerken
\usepackage{graphicx}                   % Om figuren te kunnen verwerken
\usepackage[small,bf,hang]{caption}     % Om de captions wat te verbeteren
\usepackage{xspace}                     % Magische spaties na een commando
\usepackage[latin1]{inputenc}           % Om niet ascii karakters rechtstreeks te kunnen typen
\usepackage{float}                      % Om nieuwe float environments aan te maken. Ook optie H!
\usepackage{flafter}                    % Opdat floats niet zouden voorsteken
\usepackage{listings}                   % Voor het weergeven van letterlijke text en codelistings
\usepackage[round]{natbib}              % Voor auteur-jaar citaties.
\usepackage[nottoc]{tocbibind}		% Bibliografie en inhoudsopgave in ToC; zie tocbibind.dvi
\usepackage{eurosym}                    % om het euro symbool te krijgen
\usepackage{textcomp}                   % Voor onder andere graden celsius
\usepackage{fancyhdr}                   % Voor fancy headers en footers
\usepackage[Gray,squaren,thinqspace,thinspace]{SIunits} % Om elegant eenheden zetten
% \usepackage{setspace}         % I use the command \baselinestretch to modify line spacings
% Volgend package is niet echt nodig. Het laat echter toe om gemakkelijk elektronisch
% te navigeren in je pdf-document. Deze package moet altijd als laatste ingeladen worden.
\usepackage[a4paper,plainpages=false]{hyperref}    % Om hyperlinks te hebben in het pdfdocument.


% De splitsingsuitzonderingen
\hyphenation{back-slash split-sings-uit-zon-de-ring}

%\bibpunct{(}{)}{;}{y}{,}{,}             % Auteur-jaar citaties -- zie natbib.dvi voor meer uitleg; niet echt nodig
% Het bibliografisch opmaak bestand.
% ZORG ERVOOR DAT bibliodutch.bst ZICH IN JE WERKDIRECTORY BEVINDT!!!
% \bibliographystyle{bibliodutch}

%\setlength{\parindent}{0cm}             % Inspringen van eerste lijn van paragrafen is niet gewenst.
% never use absolute sizes to modify parintents
%\renewcommand{\baselinestretch}{1.2} 	% De interlinie afstand wat vergroten.

\graphicspath{{afbeeldingen/}}               % De plaats waar latex zijn figuren gaat halen.

\makeindex                              % Om een index te genereren.

% De headers die verschijnen bovenaan de bladzijden, herdefinieren:
% deze headers mogen pas op de 2de pagina beginnen!

%\pagestyle{fancy}                       % Om aan te geven welke bladzijde stijl we gebruiken.
%\fancyhf{}                              % Resetten van al de fancy settings.
%\renewcommand{\headrulewidth}{0pt}      % Geen lijn onder de header. Zet dit op 0.4pt voor een mooie lijn.
%\fancyhf[HL]{\nouppercase{\textit{\leftmark}}} % Links in de header zetten we de leftmark,
%\fancyhead[HR]{\thepage}                % Rechts in de header het paginanummer.
% Activeer de volgende lijn en desactiveer de vorige om paginanummers onderaan gecentreerd te krijgen.
%%\fancyhf[FC]{\thepage}                  % Paginanummers onderaan gecentreerd.

% PDF specifieke opties, niet strict noodzakelijk voor een thesis.
% Is hetgeen verschijnt wanneer je in acroread de documentproperties bekijkt.
\hypersetup{
    pdfauthor = {Gaspard Lequeux},
    pdftitle = {Een Introductie tot het Zetsysteem LaTeX},
    pdfsubject = {Cursus LaTeX opgebouwd als typevoorbeeld voor het schrijven van een thesis.},
    pdfkeywords = {LaTeX, zetsysteem, thesis, eindwerk}
}


% Het volgende commando zou ervoor moeten zorgen dat er een witte ruimte wordt gelaten tussen
% elke paragraaf. Het zorgt ervoor dat er echter teveel witte ruimte komt boven en onder de
% verschillende titels, gemaakt met \section, subsection...
%%\setlength{\parskip}{0ex plus 0.3ex minus 0.3ex}

% Vandaar dat we expliciet aangeven wanneer we wensen dat een nieuwe paragraaf begint:
% \par zorgt ervoor dat er een nieuwe paragraaf begint en
% \vspace zorgt voor vertikale ruimte.
\newcommand{\npar}{\par \vspace{2.3ex plus 0.3ex minus 0.3ex}}

% Super en subscript
\newcommand{\supsc}[1]{\ensuremath{^{\text{#1}}}}   % Superscript in tekst
\newcommand{\subsc}[1]{\ensuremath{_{\text{#1}}}}   % Subscript in tekst

% Niew commando om vreemde taal weer te geven (hint: dit commando kan gebruikt
%   worden om latijnse namen, die ook cursief moeten staan, weer te geven.
\newcommand{\engels}[1]{\textit{#1}\xspace}
\newcommand{\engelsx}[1]{\index{#1}\textit{#1}\xspace}

% Niew commando om iets te benadrukken en tegelijkertijd in de index te steken.
\newcommand{\begrip}[1]{\index{#1}\textbf{#1}\xspace}

% Een nieuwe omgeving om algemene letterlijke tekst weer te geven.
\lstnewenvironment{lt}
    {
    \vspace{1.2ex plus 0.5ex minus 0.5ex}   % Beetje ruimte voor de letterlijke tekst
    \lstset{                                % Enkele opties:
        basicstyle={\small\tt},             % Iets kleiner en typmachine lettertype
        stepnumber=0,                       % De lijnen worden niet genummerd
        breaklines=true,                    % Als een lijn te lang is, wordt hij afgebroken
        basewidth={0.5em},                  % Breedte van een letter
        xleftmargin=1em}                    % Inspringing van de linker marge
    }
    {\vspace{0.9ex plus 0.5ex minus 0.5ex}  % Beetje ruimte na de letterlijke tekst
    }
% VUB templates use the verdana font. This is a sans serif font so we change the default font family to sans serif
\renewcommand{\familydefault}{\sfdefault}


\makeatletter
%\def\thickhrulefill{\leavevmode \leaders \hrule height 1pt\hfill \kern \z@}    % copied from somewhere else - no need for it and should be replaced by \newcommand...

\renewcommand{\maketitle}{
 \begin{titlepage}%

% faculteit: \small (9pt) x:41/y:25
\begin{textblock}{75.5}[0,0](41,28) % you need to keep this textblock before the VUB logo as it would otherwise be placed underneath the logo
   % \textblocklabel{Faculteit}
    \small{Faculty of Science}
\end{textblock}
% vub logo
\begin{textblock}{75}[0,0](15,12)
\textblocklabel{vub logo}
  \includegraphics[width=75.5mm]{VUB_logo.jpg}
\end{textblock}

\begin{textblock}{43}[0,0](157,229)
  \textblocklabel{vub schild}
  \includegraphics[width=43mm]{VUB_schild.png}
% Note: the color of this image should be 30% of black. I will create this image if I cannot find the command to apply alpha transparencies to an image.
\end{textblock}

% vertikale lijn: lengte: 247mm dikte hairline (0.25pt) - x: 38.5/y:25
\begin{textblock}{1}[0,0](38.5,25)
\textblocklabel{Vertical line}
    \vrule height 247mm
\end{textblock}

% eerste horizontale lijn: lengte: 161 hairline x:130/y:38.5
\begin{textblock}{161}[0,0](38.5,130)
\textblocklabel{first horizontal line}
    \hrule width 161mm
\end{textblock}

% 2de horizontale lijn: x:150/y:38.5
\begin{textblock}{161}[0,0](38.5,150)
\textblocklabel{second horizontal line}
    \hrule width 161mm
\end{textblock}

% datum: 10pt (\normalsize) - x:48/y:268
\begin{textblock}{75}[0,0](48,268)
\textblocklabel{Date}
  \normalsize{\@date}
\end{textblock}

% rapport naam: \Huge (eigelijk 30pt maar dat gaat niet) - x:48/y:100
\begin{textblock}{120}[0,1](48,123)         %  indien de titel meer dan 1 regel is schuift de titel naar boven
  \Huge{\@title}
\end{textblock}

% rapport code/ subtitle - \huge - x:48/y:137
\begin{textblock}{160}[0,0](48,133)         % use (48,133) for a 2-line title; use (48,137) for a 1-line title!
  \huge{Report project Web and Information Systems}
\end{textblock}

% auteurs \Huge - x:48/y:158
\begin{textblock}{160}[0,0](48,158)         %
  \Huge{\@author}                       %
\end{textblock}
\end{titlepage}%
\setcounter{footnote}{0}%
}

\makeatother


\author{Roeland Matthijssens, Rik Vanmechelen, Ayrton Vercruysse}
\title{WiSport}
\date{Academic year 2012-2013}

% list of possible special pages + title in english:
%\prefacename Preface 
%\refnamea References 
%\abstractname Abstract
%\bibnameb Bibliography
%\chaptername Chapter
%\appendixname Appendix
%\contentsname Contents
%\listfigurename List of Figures
%\listtablename List of Tables
%\indexname Index
%\figurename Figure
%\tablename Table
%\partname Part
%\enclname encl
%\ccname cc
%\headtoname To
%\pagename Page
%\seename see
%\alsoname see also
% if you want to change the title of these things, use something like this:
%\renewcommand*{\listfigurename}{Pictures}




%%%%%%%%%%%%%%%%%%%%%%%%%%%%%%%%%%%%%%%%%%%%%%%%%%%%%%%%%%%%%%%%%%%%%%%%%%%%%%%%%%%%%%%%%%%%%%
\begin{document}
%\frontmatter   % only for 'book', not for 'article'
\maketitle
% next 3 lines are preventing the next text from beginning on the titlepage.
% the reason is probably because I built up the titlepage with textboxes and there is 
% no normal text on it so creating a newpage doesn't work for starting on the next page.
\newpage
\vfil\null
\newpage
% everything from now on will be on the second page.
\tableofcontents
\newpage    %we want a new page after the table of content
\renewcommand{\baselinestretch}{1.5} 	% vanaf hier de interlinie afstand wat vergroten.
% regelafstand 1,5                      % indien je dit vanaf het begin doet is de titelpagina niet goed.
\small\normalsize                       % Nodig om de baselinestretch goed te krijgen.
%\mainmatter

\section{Introduction}

\paragraph{}This document will describe the implementation and usage of WiSport. WiSport is website created within the scope of the course Web and Information Systems at the Vrije Universiteit Brussel. The goal of the website is to offer its users a means to log and plan their personal training sessions. The website also allows the users to interact with others, and be social with their planned activities.
\paragraph{}WiSport offers users the possibility to create a variety of exercises. At the time of writing there are only three types of exercises (but more types can be added easily)i. The first type is an exercise based on time, e.g. walking three hours, independantly of how far one walks. The second type is one based on an amount of repetitions. e.g. Swim 10 lanes in an olimpic swimming pool. The third exercise is an exercise based on a distane. e.g. Sprinting 200 meters. Once these exercises are created, they can be shared with the WiSport community or kept private for personal use.
\paragraph{}Users can also bundle these exercises into training sessions. This allows the user to structure their workout schema in more managable chunks. Like exercises these sessions can also be shared with the WiSport community if desired.
\paragraph{}When a user finds an exercise (or training session) created by an other user that he likes he can star this exercise. This allows him to use it in his own planning as well.
\paragraph{}Once a user has created a trainings session he is allowed to "Do" the exercise. This means that he enters the information about the exercises performed into the system. The system will from then on track the progress of the user troughout all his training sessions. The user can then see his progress in clear graphs displayed with their training sessions.
\paragraph{}Of course the user doens't only want to plan what he is going to do during his workout. But he is also planning when he wants to exercise. Thats why WiSport allows the users to create events as well. These events can be given a location (where they are going to do the training) and a date. Creating these events will allow the system to notify the user of upcomming events, and generate a trainings schema displayed on a calendar. Like all resources these events can be shared, and used by other users when desired.
\paragraph{}In short WiSport is the ideal tool for anyone willing to organize their training and share their workouts with other users.
\paragraph{}They can even share their schemas with friends who are not in the WiSports community by using our restfull json export webservice

\section{Ruby On Rails}
\subsection{Framework} % (fold)
\label{sub:Framework}
Ruby on rails is an opensource full-stack webframework. This means that Ruby on Rails, or Rails in short is used to tackle every aspect of webdevelopment. Rails runs on the general purpose programming language Ruby.
The main benefit of using the Rails framework is the speed at which one can get a running web application. Almost no setup is required, and once the setup is complete, the paradigm of Convention over Configuration ensures that most of the website is developped quickly.
Ruby on rails employs, like most webframeworks, a Model-View-Controller (MVC) architecture. By default all models in the application are mapped directly to a database. This makes it easy to have persistance even when the programmer is using only the classes.
% subsection Framework (end)
\subsection{RubyGems} % (fold)
\label{sub:RubyGems}
Another huge benefit of using Rails is the fact that is has an enormous comunity. This community creates exrta plug and play functionality for a rails application by means of rubygems. Rubygems (or gems in short) are self contained libraries that can be installed into a rails application.
% subsection RubyGems (end)


\section{Used API's}
\subsection{Gravatar}
\paragraph{}It is always nice to have an avatar to users. The web service Gravatar gives users a central location to setup avatars. We use the Gravatar API to automatically take the user's gravatar and display it on his profile page.
\subsection{Authentication}
\paragraph{}Users often don't like to create yet an other account, with a username and password which they will not remember. This is why we decide to allow users to login using different web services. The Omniauth gem \footnote{Visit https://github.com/intridea/omniauth for more information} makes it easy to add authentication from some of the major services on the web. Our most used log in method is Google. It will take your Google plus profile picture and use it on the website, in stead of the Gravatar avatar.
\subsection{Weather}
\paragraph{}In class we used wunderground in the exercises. Their easy API made us chose them above weather.com.
\subsection{Maps}
\paragraph{}Google maps is integrated using the google maps for rails gem\footnote{Visit https://github.com/apneadiving/Google-Maps-for-Rails for more information}. This gem provides us hooks between the rails controller and the Google maps Javascript API. We use html5 geolocation for location based queries, as well as IP based location. You can read more about IP based location in the next section
\subsection{IP based location}
\paragraph{}In a many cases it is not necessary to annoy the user with permission requests, that is why we decided to also use IP based location. InfoDB\footnote{http://ipinfodb.com/} provides you with an easy API to translate IP addresses into geographical coordinates.
\subsection{Media hosting}
\paragraph{}Users can add pictures and movies to exercises. In stead of storing the pictures on our own servers, we use the Imgur api\footnote{Visit http://api.imgur.com/ for more information} to upload the pictures to Imgur. Currently we only accept youtube clips. These will be displayed using the static youtube API.
\subsection{Graphs}
\paragraph{}For displaying both system wide statistics, as well as userbased statistics we use the HighChart API. This is a JavaScript only API that can be loaded in a web page. After that we comunicate with the plugin through XML configurations.

\section{Features}
\subsection{Login through social media}
\paragraph{}Next to the tradtional sign up and login with an account on our website we offer the possibility to log in with most of the popular social media profiles. We allow loggin in  with a Google, Facebook or Twitter account. When logging in with your Google account the picture of you Google profile is used on to your WiSport profile.

\subsection{Editting sessions/exercises/events}
\paragraph{}After you have created an exercise, session or event you have to possibility to change this event. In the case of an exercise you can edit your properties of this exercise. A clear example of this could be if you have a distance exercise, in wich you run 5 kilometers. Odds are that after six months you don't even bother running 5 kilometers, but would like to start running 10 kilometers instead.
\paragraph{}However you already have a whole graph of your progress of this exercise, and you might want to keep tracking your progress through both 5 and 10 kilometers run. If this is the case you don't need to create a new exercise but you can edit the existing exercise to a 10 km running exercise instead. As the graphs of your progress do not rely on this value, but rely on the value you enter when doing an exercise this will not affect your progress on this exercise.
\subsection{Publish exercises and sessions}
\paragraph{}When a user creates an exercise or session and they are pleased with it they can share it whith the WiSport community. This is done by pressing the Publish button showed on each page of an exercise or training session, showed within the list of own exercises, or even when creating an exercise or session you can immediatly choose to publish this exercise or session.
\paragraph{}The published exercises can be starred by other users. As explained in the starred feature when starred a user can use the exercise or session to compose their  own sessions or events.
\subsection{Smart searchfields for exercises and sessions}
\paragraph{}Within the list of all exercises and session search form are implemented. You can search by name of the exercise/session, the creator, or the description. Beside this we also implemented a select box for the exercises. With this select box you can choose the type of exercises that are displayed. This can help you whenever you are looking for a particular exercise to add to you own exercises. All these search forms can be combined together so that you can narrow your search down to your needs.
\paragraph{}The search forms are AJAX forms, this means you don't need to press search anywhere, but they dynamically change the list filtering out any results not matching the searchfield.

\subsection{Star sessions/exercises}
\paragraph{}Every user has the possibility to publish their exercises or sessions. When published these will apprear in the list of all exercises or sessions and other users will be able to see these. When a user is browsing through this list they can star exercises or session which look particular intresting to them. Starred items will also apear in their personal exercises or sessions.
\paragraph{}This means that starred exercises can be used to create sessions and starred session can be used to organise an event containing these sessions.

\subsection{Do a session}
\paragraph{}When a session, or event, is created you still need to do the session. This means that even though you planned an event today, you still have to say on the website that you did the exercises specified in the session. When entering the session in the system you are allowed to enter some extra information about the performed exercises. This can be some additional notes, but, more importantly you can insert the actual values you did for each exercise. This means that if your exercise told you to run 10 kilometers, but you only ran 8 you can enter this in the systems and this will will be taken into account when creating your progression graph.
\subsection{Graphs tracking progress}
\paragraph{}After you have completed a session and entered the actual numbers you performed to each exercise a graph will be generated for earch exercise within this session. This gives you the possibility to track your progression on a session, and even more you can check the number of times you completed an exercise, or the entire session.
\paragraph{}On the homepage there are also some graphs showing some system wide information. There is a user-activity graph, showing how much exercises users are doing currently. A piechart is also displayed showing the popularity of the different types of exercises.
\subsection{Calendar view for events}
\paragraph{}For events we make use of a clear calendar view. On this view you can see your planned events, and from this view you can go to the actual page of the events, delete or edit events. This gives you a clear view on your workload and makes it easier to plan your sessions

\subsection{Weather}
\paragraph{}On the events page before creating an event you can check the weather for your current location. This gives you the possibility to plan your exercise session on a day the weather is suitable. For example shedule your indoors exercises on a rainy day and plan your outdoor activities on less rainy days
\subsection{Share on Google+}
\paragraph{}When you finished a nice exercise or session of which you are really proud you can share this on your Google+ profile. We have chosen Google+ because only Google+ ensures you that you only share this information with people you want it to be shared with. The is for privacy reasons, especially sharing on the internet when you are not going to be home isn't always a good idea.

\subsection{Gravatar usage}
\paragraph{}When displaying the user's profile page, we try to show their profile image as well. When a google account's profile picture is displayed we try to get the information from google. However when this information is not available, we request an image from the popular avatar website Gravatar. If this does not yield a match, we will display a placeholder image as a backup plan.
\subsection{Scaleable interface}
\paragraph{}Thanks to the use of Twitter Bootstrap the entire interface scales to the size of the screen it is displayed on. This means that when the size of the screen is made smaller the entire website is scaled with it. Even when the size is to small to show the entire header line this line is reformed to a dropdown box. This is particulary handy for rendering the website on mobile devices, or devices with smaller screens.
\subsection{Add location to an event by giving address}
\paragraph{}Whenever an event is created the user can enter where this event will take place. This is done by just entering the address of the location or even keywords (we will request the propper location from google). When the event is created a map will be displayed which shows the location of the event.
\subsection{Location my ip/ geoloc}
\paragraph{}For the usage of the Weather API we try to get the user's location by geolocation. However when this fails we make use of Location by ip API. This is however only a backup plan, since the IP based location is not as accurate on mobile devices
\subsection{Find nearby events}
\paragraph{}When going to the events page your can find all events planned on a map. This map is centralized around your current location, so the closest events will be shown first. This gives you the ability to find all near events and if you so desire, you can tag along on these events.
\paragraph{}By clicking on the markers more information is shown about this event. You can also go to the event page to get more information about the event, and go along with this event.
\subsection{Database populator for testing purpose}
\paragraph{}For testing purposes a database populator is created. This creator generates random data, such as users, exercises, training sessions and events.
\paragraph{}The use of this populator is to check the site's performances when handeling a higher number of users and data. This gives us the possibility to do some experiments with a lot of data. Especially for visualising the website when a lot of data is added to the website. These scripts allowed us to verify the scalability of the website's design
\section{Tutorial}
\subsection{Logging in}
\paragraph{}When entering the website for the first time you'll get some general information about the website. The first thing we advice you to do is create an account, by pressing the Sign in button. Once you've made an account, or want to use one of your social media accounts you can log in by pressing the Log In button and choose whether to enter the account information of an account created on our website, or log in trough Google, Facebook or Twitter.
\paragraph{}Once logged in you reach a new screen with some general information about your account. These are things like upcomming events, recently passed events and some statistics about the usage of your account.

\subsection{Exercise}
\subsubsection{My Exercises}
\paragraph{}When clicking under the tab Exercises on My Exercises you'll see a list of all exercises you made and all exercises you starred. The exercises in this list can be used to later on make some training sessions.
\paragraph{}For every exercise the you created you'll get some options as Destroy, Edit or Publish/Unpublish. The Destroy will delete this exercise from the system, Edit wil give you ability to edit this exercise and Publish/Unpublish decides whether this exercise will show up in the list of all exercises.
\subsubsection{All Exercises}
\paragraph{}Under the tab All Exercises you'll see all exercises made by other users which are Published. You'll be able to take a look at the page of the exercise, to see what this exercise does by just clicking on it's name. You'll have to possibility to check the creators page by clicking on the name of the creator.
\paragraph{}In front of the exercises in this list stands a little star which can be toggled on or off. When this star is toggled on, when the exercise is starred, this exercise will appear in the My Exercises list.
\paragraph{}As you can see underneath the Title and Owner page there are some searchfields. Here you can enter some letters and instantly the list of exercises will be filtered, only containing those titles or users containing this sequence of letters. Under the Type you'll see a foldbox in which you can enter the type of exercises you are looking for.

\subsubsection{Creating an exercise}
\paragraph{}From within the My Exercises page you'll be able to press the New button to create a new exercise. Creating a new exercise will ask you to give a name to the exercise, a type, decide if you will publish it or keep it private (which can be editted later on), it will also give you the chance to give a small description with this exercise.
\paragraph{}At the bottem you'll be able to add a link to a youtube video, what can for example give a demo of the exercise, or add one or more pictures bu or dragging them into the drag box, or by selecting them using the Select File button. 
\paragraph{}To create the exercise you simply press Create Exercise on the bottem, or press Cancel if you changed your mind.
\paragraph{}When pressing the Edit button of an exercise you'll get a similar form again, with the old information filled in. This will give you the possiblity the edit this information.

\subsubsection{The exercise page}
\paragraph{}Each exercise has a page of it's own. On this page you can see more information about the exercise, such as the added picturer, or added youtube video's, the value's entered for each sort of exercise (for example 10 km) and you can read the entire description of the exercise.
\paragraph{}As owner of an exercise you'll be provided by some extra button, where you can choose to Edit, Publish/Unpublish the exercise or Destroy the exercise. When looking at an exercise that is not made by you you'll see who created this exercise.

\subsection{Training sessions}
\subsubsection{My Training sessions}
\paragraph{}In the My Training sessions page every session made by you, or starred by you will be listed. This list can be seen as the All Exercises page. It contains search fields as the exercises, but here you can search on the title, the creator and the desciption of a session. Also searching by exercises contained by a session can be done.
\paragraph{}As an extra button here you can Do a training session. What this does you can check in the Do a trainings session section. When pressing on the descirption of the training session you can see what exercises this training session contains.
\subsubsection{All Training sessions}
The All trainings Sessions can be seen as the All Exercises page. The differences are offcourse the search fields, here you can search, similar to the My Trainings Session on owner, name and descirption. Similar as with the My Training Session page you can click on the discription to show the exercises it contains.

\subsubsection{Create Training session}
\paragraph{}By pressing the New button on the My Training Session page you'll be able to create a new trainings session. To create a new training session you'll need to choose a number of exercises. You will be able to choose between all exercises listed in your My Exercise page. You'll also be able to add a little description about the session and, as with the exercises you'll be able to publish these instantly.
\subsubsection{The Training session}
\paragraph{}When clicking on the name of a training session you'll be redirected to this page. This page contains all information about a training session such as the exercises that need to be done within this session and a description of this session. At the bottem of this page you'll see some graphs containing your progress on each exercise of this session. With this you'll be able to track you progress on this exercise.
\subsubsection{Do a Training session} 
\paragraph{}When you're at the My Training Sessions page, the All Trainings session page or at the page of a trainings session you'll see a Do button This buttons allows you to tell the system you've done the session and submit your results. You can give it a small note, you can enter the real value's you've realized (when you for example only ran 8km on a 10km exercerise) and enter the time you needed to complete an exercise. This data will be used to create the graphs showing your progress.

\subsection{Events} 
\paragraph{}When you click on the Events link you'll find your personal calendar within Wisport. On this calendar you can see you planned events in the future or in the past. Next to this you can remove an event, or edit an event by clicking on the buttons next to the eventnames on the calendar.
\paragraph{}On top of the page you'll see a Get Weather Forecast button, this will show you the weather for the comming week, based on the location recieved by your ip-address. This can be a usefull help to plan outdoor events on days it's less likely to rain.
\subsubsection{Create an event}
\paragraph{}On the events page you'll see a new button to create a new event. When creating a new event you'll need to choose the training session you want to do during this event and pick a date on which you want to do it using the date picker. You'll also be able to add some extra information about this event and give the address of the event. This address will later, on the events page be shown on a map.
\paragraph{}When you've entered all information about the event you'll simple click in the Create event button at the bottem or the cancel bottem if you've changed your mind.
\paragraph{}Whenever you'll click on the edit button of an event you'll be redirected to this this sort of form with the old information filled in so you can simple edit the information.
\subsubsection{Events page}
\paragraph{}When clicking on an event on the calendar the events page will be shown. This page will show the location of the event on a map, will show the description of the event and at the bottem it will show the exercises contained by the session of this event.
\paragraph{}As owner of the event you'll also be able to see the Edit and Destroy buttons on the top right of the events page to respectivily edit or destroy this event.
\subsubsection{Nearby events}
\paragraph{}When pressing the nearby events button you'll get a map displaying all events on map. The map will be centered around your current location what will make you see the nearest events first. When cliking on a marker of an event you'll see the name and descirption of the event and you'll be able to follow the link to the event page of this event.
\subsection{Profile}
\paragraph{}When logged in you can enter your profile page by clicking on your name in the topright corner. This page will show you your own profile, an will give you to opportunity to edit some profile information. Next to the profile information are you upcomming and recently passed events showed here.
\paragraph{}At the bottom a list of your own exercises and sessions is shown, aswell as the exercises and session shown by people you follow.
\paragraph{}As an extra bit of information there is shown a how many people that are following you.
\paragraph{}When pressing on the name of a user, in for example the exercises or sessions list brings you to the profile page of this person. This page gives you the possibitly to follow this person, so you can track the exercises and sessions he made.
\section{Deployment information}
\subsection{Deploy for Development}
When you use ruby a lot, it is recommended that you use something like the Ruby Version Manager\footnote{Visit https://rvm.io for more information}, or RVM for short. This will help you manage different ruby versions, and different GEM sets. RVM allows you to create separate sets of gems for each project you do. We will assume you have this setup or, in case you do not want to use rvm, know that it can interfere with other ruby projects.\\
Out of experience, there is the possibility that your system does not include some of the libraries needed to install all of the gems used in this project. The most common are: libxslt\footnote{Visit http://nokogiri.org/tutorials/installing_nokogiri.html for more information} and nodeJS \footnote{Visit http://nodejs.org/ for more information}. Make sure these are installed before you start.
To start the development on this project you will first have to check out the git repository using the following commando.
\begin{lstlisting}
git clone git://github.com/rikvanmechelen/web_info_sys_sport.git
\end{lstlisting}
This will create the project directoy. Now go into the wisport directory, which consequetly is the \textit{RAILS\_ ROOT}, the root of the Ruby on Rails project.\\
Since we use SQLite as DBMS, all you have to do is run:
\begin{lstlisting}
	rake db:migrate
\end{lstlisting}
Once this is done, you are free to poppulate the database: 
\begin{lstlisting}
	rake db:populate[10]
\end{lstlisting}
the 10 means that we will create 10 users.\\
By now you have a fully working rails application.
To start the rails server do:
\begin{lstlisting}
	rails server
\end{lstlisting}
All that rests, is pointing your favourite web browser to http://localhost:3000

\subsection{Deploy for production}


% De bibliografie en de index
%\backmatter
\newpage        % we want the bibliography to start at a new page
\bibliography{bibliografie}
\newpage
\printindex                             % Om de index af te printen, niet bij een thesis.

\end{document} 
